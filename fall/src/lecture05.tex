\documentclass[aspectration=1610,t]{beamer}
\usepackage{csc}
\title{Лекция 5. Буфер кадра}


\date{
   \textbf{ИТМО}\\
   12 октября 2022\\
   Санкт-Петербург
}

\begin{document}

\begin{frame}
  \titlepage
\end{frame}

\begin{frame}[fragile]{Буфер кадра}
    Все операции требуют {\bf frame buffer}. Этот буфер может содержать несколько компонент ({\bf attachments}).
    \begin{itemize}
        \item {\bf color} - цвет;
        \item {\bf depth} - глубина;
        \item {\bf stencil} - трафарет.
    \end{itemize}
\end{frame}

\begin{frame}[fragile]{FBO}
   Создание и использование буфера:
            {\small \begin{lstlisting}
int fbo = -1;
glGenFramebuffers(1, &fbo);

glBindFramebuffer(GL_FRAMEBUFFER, fbo);

// Do smth

glBindFramebuffer(GL_FRAMEBUFFER, 0);
glDeleteFramebuffers(1, &fbo);
            \end{lstlisting}}
    Бывают другие типы FBO (read-only, write-only, read-write).
\end{frame}

\begin{frame}[fragile]{FBO}
    Для использования буфера:
    \begin{itemize}
        \item Должен быть подключен как минимум один буфер (цвета, глубины или трафарета).
        \item Должно присутствовать хотя бы одно прикрепление цвета (color attachment).
        \item Все подключения также должны быть завершенными (обеспечены выделенной памятью).
        \item Каждый буфер должен иметь одинаковое количество семплов.
    \end{itemize}
    Для проверки статуса FBO используют следующую конструкцию:
    {\small \begin{lstlisting}
glCheckFramebufferStatus(GL_FRAMEBUFFER)
    == GL_FRAMEBUFFER_COMPLETE
    \end{lstlisting}}
\end{frame}

\begin{frame}[fragile]{Attachments}
    Типы прикреплений.
    \begin{itemize}
        \item {\bf texture} - текстура;
        \item {\bf renderbuffer} - рендер-буфер.
    \end{itemize}
\end{frame}

\begin{frame}[fragile]{Texture attachment}
    Создаем текстуру:
    {\small \begin{lstlisting}
int texture;
glGenTextures(1, &texture);
glBindTexture(GL_TEXTURE_2D, texture);
glTexImage2D(GL_TEXTURE_2D, 0, GL_RGB, 
    800, 600, 0, GL_RGB, GL_UNSIGNED_BYTE, NULL);
glTexParameteri(GL_TEXTURE_2D, 
    GL_TEXTURE_MIN_FILTER, GL_LINEAR);
glTexParameteri(GL_TEXTURE_2D,
    GL_TEXTURE_MAG_FILTER, GL_LINEAR);  
    \end{lstlisting}}
    Прикрепляем текстуру:
    {\small \begin{lstlisting}
glFramebufferTexture2D(GL_FRAMEBUFFER, 
    GL_COLOR_ATTACHMENT0, GL_TEXTURE_2D, texture, 0);  
    \end{lstlisting}}
\end{frame}

\begin{frame}[fragile]{Attachments}
    Можно прикреплять несколько буферов цвета.
    \begin{itemize}
        \item {\bf GL\_COLOR\_ATTACHMENT0};
        \item {\bf GL\_COLOR\_ATTACHMENT1};
        \item {\bf ...}
    \end{itemize}
    Аналогичным образом можно прикреплять текстуры для буферов глубины и трафарета.
    \begin{itemize}
        \item {\bf GL\_DEPTH\_ATTACHMENT};
        \item {\bf GL\_STENCIL\_ATTACHMENT};
        \item {\bf GL\_DEPTH\_STENCIL\_ATTACHMENT};
    \end{itemize}
\end{frame}

\begin{frame}[fragile]{Texture attachment}
    Пример комбинированного буфера:
    {\small \begin{lstlisting}
glTexImage2D(
    GL_TEXTURE_2D, 0, GL_DEPTH24_STENCIL8, 
    800, 600, 0, GL_DEPTH_STENCIL, 
    GL_UNSIGNED_INT_24_8, NULL);
    
glFramebufferTexture2D(GL_FRAMEBUFFER, 
    GL_DEPTH_STENCIL_ATTACHMENT, 
    GL_TEXTURE_2D, texture, 0);          
    \end{lstlisting}}
\end{frame}

\begin{frame}[fragile]{Renderbuffer attachment}
    Создание и установка:
    {\small \begin{lstlisting}
int rbo;
glGenRenderbuffers(1, &rbo);
glBindRenderbuffer(GL_RENDERBUFFER, rbo); 
glRenderbufferStorage(GL_RENDERBUFFER, 
    GL_DEPTH24_STENCIL8, 800, 600);        
    \end{lstlisting}}
    Подключение:
    {\small \begin{lstlisting}
glFramebufferRenderbuffer(GL_FRAMEBUFFER, 
    GL_DEPTH_STENCIL_ATTACHMENT, GL_RENDERBUFFER, rbo); 
    \end{lstlisting}}
\end{frame}

\end{document}

